\documentclass[a4paper]{article}

\usepackage{CJK}
\begin{CJK*}{UTF8}{gbsn}

\begin{document}

\title{Title}
\author{邱铭达 1000010310}
\date{Today}
%\maketitle

%
% Title Page
%
\begin{center}
	\Huge \textbf{Title}
\end{center}
\vspace{1 in}
\begin{center}
	\normalsize 邱铭达(1000010310) 吕一鼎(1000010)
\end{center}
\newpage


\tableofcontents

\section{问题}
\begin{equation}
	\overtilde(U)
\end{equation}

\subsection{LC振荡电路}
	LC振荡电路是物理学、电子学中应用广泛的一类电路设计。其广泛应用于电子钟、电磁炉、无损检测等现代领域,发挥着巨大的作用。
\subsection{基本模型}
	在现实中,由于LC振荡电路不可能做到无阻尼振荡,所以一般会添加三极管等原件作为放大器,此类原件在理论分析中意义不大,所以不予考虑。经过简化,最基本的LC振荡电路如下图所示: 图中基本的电路元件是一个电感和电容,为了模拟实际情况,所以加上一个电阻R。电源接通后,电路呈现振荡状态,分为电容的充电过程:电场能增加,磁场能减小,回路中电流减小,电容器上电量增加——即磁场能在向电场能转化;以及电容的放电过程:电场能减少,磁场能增加,回路中电流增加,电容器上的电量减少——即电场能在向磁场能转化。
\subsection{模型的数学描述}

\subsection{目的}

	我们的目的即是要解决这样一个黑箱问题:给定一个基本的LC振荡电路,我们除了电势差E以外不知道它的各项参数(电容、电感以及电阻);而现在测量得知一批关于其振荡过程的电压、电流数据,我们意图利用这些数据进行曲线拟合,以得出电路中各个元件的信息,包括电容的值、电感的值和电阻的值。

\section{选用算法}

	考虑到本问题实际上是一个曲线拟合问题,所以我们决定使用基因遗传算法。

\subsection{基因遗传算法-简介}

	遗传算法(Genetic Algorithm)是一类借鉴生物界的进化规律(适者生存,优胜劣汰遗传机制)演化而来的随机化搜索方法。它是由美国的J.Holland教授1975年首先提出,其主要特点是直接对结构对象进行操作,不存在求导和函数连续性的限定;具有内在的隐并行性和更好的全局寻优能力;采用概率化的寻优方法,能自动获取和指导优化的搜索空间,自适应地调整搜索方向,不需要确定的规则。遗传算法的这些性质,已被人们广泛地应用于组合优化、机器学习、信号处理、自适应控制和人工生命等领域。它是现代有关智能计算中的关键技术。

\subsection{基因遗传算法-基本原理}
遗传算法是从代表问题可能潜在的解集的一个种群(population)开始的,而一个种群则由经过基因(gene)编码的一定数目的个体 (individual)组成。
每个个体实际上是染色体(chromosome)带有特征的实体。染色体作为遗传物质的主要载体,即多个基因的集合,其内部表现(即基因型)是某种基因组合,
它决定了个体的形状的外部表现,如黑头发的特征是由染色体中控制这一特征的某种基因组合决定的。因此,在一开始需要实现从表现型到基因型的映射即编码工作。由于仿照基因编码的工作很复杂,我们往往进行简化,如二进制编码,初代种群产生之后,按照适者生存和优胜劣汰的原理,逐代(generation)演化产生出越来越好的近似解,在每一代,根据问题域中个体的适应度(fitness)大小选择(selection)个体,并借助于自然遗传学的遗传算子(genetic operators)进行组合交叉(crossover)和变异(mutation),产生出代表新的解集的种群。这个过程将导致种群像自然进化一样的后生代种群比前代更加适应于环境,末代种群中的最优个体经过解码(decoding),可以作为问题近似最优解。

\section{}


\end{CJK*}
\end{document}
