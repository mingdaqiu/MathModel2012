\documentclass[a4paper]{article}

\usepackage{CJK}
\begin{CJK*}{UTF8}{gbsn}

\begin{document}

\title{Title}
\author{邱铭达 1000010310}
\date{Today}
%\maketitle

%
% Title Page
%
\begin{center}
	\Huge \textbf{Title}
\end{center}
\vspace{1 in}
\begin{center}
	\normalsize 邱铭达(1000010310) 吕一鼎(1000010679)
\end{center}
\newpage


\tableofcontents

% 电路上的需求: 猜测电路的属性
\section{题目的产生}
在现实生活中,电路问题是一种非常重要的问题,同时它们也有着非常广泛的实际应用。其中的一类就是黑性问题,即根据某些观测值确定电路的基本构成和电学原件的值。对于直流稳恒电路来说,这是一个较为简单的问题,只要根据测量值进行一些简单的计算即可得到电路的组成情况;然而,对于含有电感、电容的非稳恒直流电路或者干脆就是交流电路来说,由于电路中存在振荡过程,一方面由于观测值存在误差,另一方面振荡过程本身不易计算,使得根据测量值来估计黑箱电路情况变得相对困难。基于这一现实情况,我们提出了这一课题——使用曲线拟合来根据少量的观测值模拟振荡电路组成状况。
%
\section{使用遗传算法进行曲线拟合}
\subsection{遗传算法的原理}
\subsection{遗传算法的实现}
\subsection{遗传算法的曲线拟合算法}

\section{具体的电路问题的解决}
由于实际的电路问题五花八门,种类丰富,所以我们仅取其中比较有代表性的两种进行分析:其一是直流情况下的RLC振荡电路,它的振荡是一个暂态过程;其二是RLC谐振电路,它是一个稳定的振荡电路,使用交流源。
\subsection{RLC-直流振荡电路}
RLC直流振荡电路是很基本的一种直流振荡电路,三极管放大器+LC振荡电路的组合是诸多家用电器的基本设计思想,比如电磁炉、电子钟等。
为使模型简化处理,我们选用了一下这张电路图:
插图
在以下分析中,E表示电源电动势,R表示串联的电阻,C表示电容,L表示电感,U_L表示电感的电压,U_C表示电容的电压,q表示电容器上的电荷,I_L表示流经电感的电流瞬时值,I_C表示流经电容的电流瞬时值。
对于电感有自感公式:
\begin{displaymath}
U_L=L\frac(dI_L)(dt)
\end{displaymath}
对于电容有:
\begin{displaymath}
I_C=\frac{dq}{dt}
q=U_CC
\end{displaymath}
对于电阻有:
\begin{displaymath}
U_R=IR
\end{displaymath}
此时根据基尔霍夫定律有:
\begin{displaymath}
U_C+U_R=E
I_C+I_L=I
\end{displaymath}
联立得到关于U_C的微分方程:
\begin{displaymath}
\frac{d^{2}U_C}{dt^{2}}+\frac{1}{RC}\frac{dU_C}{dt}+\frac{1}{LC}U_C=0
\end{displaymath}
通过微分方程数值解可以得到该方程在一定L、一定C下的图像,下面取一组数据E=10V,R=10\Omega,L=0.1h,C=0.1F得到模拟的观测值以后进行测试
模拟的观测值
插图
根据观测值拟合的图像
插图
拟合图像同实际图像对比
插图
如图所示,使用基因算法拟合出的曲线与实际曲线符合得很好。
\subsection{RLC-谐振电路}
RLC谐振电路是另一类应用广泛的振荡电路,其在谐振频率振荡幅度大、偏离谐振频率震荡幅度大幅减小的特性使得此类电路大量应用于电子滤波、选频的场合,如收音机、无线电收发器等。
如下图所示,这是一张基本的串联LC谐振电路;我们的目的就是在不知道电路L、C值的情况下,通过有限的I与角频率\omega 观测值,使用线性拟合的办法找出电路的谐振频率以及电感、电容的值。
插图
由于上图为交变电路,所以选用交变电路的复数方法进行分析:
公式分析
在实际生活中,以收音机为例,它的调频范围一般是535~1600kHz,所用电源一般为两节5号电池。所以在以下测试中,取电源为3V交流源,调频范围600kHz~1600kHz,先取C=10^{-6},L=10^{-6}来得到观测值。
模拟的观测值
插图
根据观测值拟合的图像
插图
拟合图像同实际图像对比
插图

\section{总结}





\end{CJK*}
\end{document}
